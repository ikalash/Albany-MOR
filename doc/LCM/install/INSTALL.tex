\documentclass{article}
\usepackage[utf8]{inputenc}
\usepackage{amsmath}
\usepackage{amsfonts}
\usepackage{amssymb}
\usepackage{graphicx}
\usepackage{fancyvrb}
\usepackage{color}
\usepackage{booktabs}
\usepackage{float}
\usepackage{supertabular}
\usepackage[left=2cm,right=2cm,top=2cm,bottom=2cm]{geometry}

\newcommand{\trilinos}{\textsc{Trilinos}}
\newcommand{\albany}{\textsc{Albany}}
\newcommand{\lcm}{\textsc{LCM}}
\newcommand{\kokkos}{\textsc{Kokkos}}

\let\oldv\verbatim \let\oldendv\endverbatim

\def\verbatim
{\par\setbox0\vbox\bgroup\oldv}

\def\endverbatim
{\oldendv\egroup\fboxsep0pt \noindent\colorbox[gray]{0.95}{\usebox0}\par}

\title{Installing and compiling \albany{}/\lcm{} and
  \trilinos{}}

\author{The \albany{}/\lcm{} team\thanks{Original version by Juli\'an
    R\'imoli}}

\begin{document}

\maketitle

\section{Introduction}
This document describes the necessary steps to install \albany{}/\lcm{} and
\trilinos{} on:
\begin{itemize}
\item \verb+Fedora Linux 20-27+
\item \verb+Ubuntu Linux 14.04 LTS, 16.04LTS, and 18.04 LTS+
\item \verb+Scientific Linux 7.5+
\item \verb+The Linux CEE/SEMS environment at Sandia+
\item \verb+Sandia's HPC systems Chama and Skybridge+
\end{itemize}

\section{Required Packages for Sandia's CEE and HPC}
The Linux CEE/SEMS environment and the HPC clusters already have all the
required libraries installed. Skip to Section \ref{sec:repo} to set up the
source code.

\section{Required Packages for Fedora and Scientific Linux}
For Scientific Linux, install first the following package
\begin{verbatim}
yum install epel-release
\end{verbatim}
The following packages should be installed using the \verb+dnf install+ command
on Fedora and the \verb+yum install+ command on Scientific Linux
\begin{verbatim}
blas
blas-devel
boost
boost-devel
boost-openmpi
boost-openmpi-devel
boost-static
cmake
hdf5
hdf5-devel
hdf5-openmpi
hdf5-openmpi-devel
hdf5-static
lapack
lapack-devel
\end{verbatim}

\begin{verbatim}
matio
matio-devel
netcdf
netcdf-devel
netcdf-openmpi
netcdf-openmpi-devel
netcdf-static
openmpi
openmpi-devel
environment-modules
gcc-c++
git
\end{verbatim}

For example, to install the first package you should type
\begin{verbatim}
sudo dnf install blas
\end{verbatim}

Make sure that all these packages are installed, specially if you
create a script to do so. If a package is not installed because of a
typo then the compilation will fail.

It may be necessary to logout and login for the module alias from the
\verb+environment-modules+ package to become active.

Optional but strongly recommended packages:
\begin{verbatim}
clang
clang-devel
gitk
\end{verbatim}

\section{Required Packages for Ubuntu}
The following packages should be installed using the \verb+apt install+
command
\begin{verbatim}
libblas3
libblas-dev
libboost-dev
libboost-graph1.54
libboost-program-options1.54
cmake
libhdf5-openmpi-7
libhdf5-openmpi-dev
liblapack3
liblapack-dev
libnetcdf-c++4
libnetcdf-dev
libopenmpi1.6
libopenmpi-dev
mpi-default-bin
environment-modules
g++
gfortran
git
\end{verbatim}

Depending on the Ubuntu version, the package version may differ slightly, for
example:

\begin{table}[H]
  \begin{center}
    \begin{tabular}{l l l}
      \toprule
      \verb+Ubuntu 14.04 LTS+ &
      \verb+Ubuntu 16.04 LTS+ &
      \verb+Ubuntu 18.04 LTS+ \\
      \midrule
      \verb+libboost-graph1.54+ &
      \verb+libboost-graph1.58+ &
      \verb+libboost-graph1.65+ \\
      \verb+libboost-program-options1.54+ &
      \verb+libboost-program-options1.58+ &
      \verb+libboost-program-options1.65+ \\
      \verb+libopenmpi1.6+ &
      \verb+libopenmpi1.10+ &
      \verb+libopenmpi100+ \\
      \bottomrule
    \end{tabular}
  \end{center}
\end{table}

To install the
first package you should type
\begin{verbatim}
sudo apt install libblas3
\end{verbatim}

Make sure that all these packages are installed, specially if you
create a script to do so. If a package is not installed because of a
typo then the compilation will fail.

It may be necessary to logout and login for the module alias from the
\verb+environment-modules+ package to become active.

\section{Repository Setup with GitHub}
\label{sec:repo}

In a web browser go to \verb+www.github.com+, create an account and
set up ssh public keys. If you require push privilges for \albany{},
email Glen Hansen at \verb+gahanse@sandia.gov+ and let him know
that. On the other hand, if you require push privileges for \trilinos{},
it is best if you contact the \trilinos{} developers directly. Go to
\verb+www.trilinos.org+ for more information.

It is strongly recommended that you join the \verb+AlbanyLCM+ Google
group to receive commit notices. Go to
\verb+groups.google.com/forum/#!forum/albanylcm+ and join the
group. You can also browse the source code at
\verb+github.com/gahansen/Albany+.

\section{Directory Structure}
In your home directory, create a directory with the name \verb+LCM+:
\begin{verbatim}
mkdir LCM
\end{verbatim}

Change directory to the newly created one:
\begin{verbatim}
cd LCM
\end{verbatim}

Check out the latest version of \trilinos{}, which is hosted now on
GitHub:
\begin{verbatim}
git clone git@github.com:trilinos/Trilinos.git Trilinos
\end{verbatim}

Finally, check out the latest version of \albany{}:
\begin{verbatim}
git clone git@github.com:gahansen/Albany.git Albany
\end{verbatim}

At this point, the directory structure should look like this:
\begin{verbatim}
LCM
  |- Albany
  |- Trilinos
\end{verbatim}

\section{Environment Variables}
In \verb+~/.bashrc+, the following variables are needed:
\begin{verbatim}
export LCM_DIR=~/LCM
module use --append $LCM_DIR/Albany/doc/LCM/modulefiles
\end{verbatim}
The \verb+LCM_DIR+ variable should contain the location of the
top-level \verb+LCM+ directory.

For Ubuntu and Scientific Linux, it is also necessary to add to \verb+~/.bashrc+
the following definitions before the previous ones:
\begin{verbatim}
modules_shell=bash
module() { eval `/usr/bin/modulecmd $modules_shell $*`; }
\end{verbatim}

\section{Build Script}
Create a symbolic link to the build script
\verb+LCM/Albany/doc/LCM/build/build.sh+ to the top-level \verb+LCM+
directory. Make sure that the build script is executable and
read only:
\begin{verbatim}
cd ~/LCM
chmod 0555 build.sh
\end{verbatim}

The \verb+build.sh+ script performs different actions according to the name with
which it is invoked. This is accomplshed by creating symlinks to \verb+build.sh+
and using them to run it. For example:
\begin{itemize}
\item \verb+clean.sh+ will delete all traces of the corresponding build and will
  create a new configuration script based on the corresponding template.

\item \verb+config.sh+ will attempt to configure the build.

\item \verb+build.sh+ (original name) will build using cmake.

\item \verb+test.sh+ will run the cmake tests.

\item \verb+update.sh+ will execute git pull in the package repository, and if
  combined with dash below, it will send a report of changed files to CDash.

\item \verb+dash.sh+ will post the results of ctest to configured CDash site.

\item symlinks with combinations of the above
  (e.g. \verb+clean-config-build.sh+) will perform the specified actions in
  sequence. See \verb+build.sh+ for valid sequences.
\end{itemize}
For example, the following symbolic links will create separate
commands for clean up, configuring and testing:
\begin{verbatim}
ln -s build.sh clean.sh
ln -s build.sh config.sh
ln -s build.sh test.sh
\end{verbatim}
They could also be combined for convenience:
\begin{verbatim}
ln -s build.sh clean-config.sh
ln -s build.sh clean-config-build.sh
ln -s build.sh clean-config-build-test.sh
ln -s build.sh config-build.sh
ln -s build.sh config-build-test.sh
\end{verbatim}
There is also a script
\verb+LCM/Albany/doc/LCM/install/albany-lcm-symlinks.sh+
that will create the appropriate symbolic links.

The build system is based on CMake. Thus the ouput verbosity level can be
controlled by passing \verb+-V+ or \verb+-VV+ as a final option to
\verb+build.sh+ or its aliases.

\section{Parallel Schwarz and DTK}
\label{sec:dtk}

This section applies only if using the Schwarz alternating method in parallel by
means of the Data Transfer Kit (\verb+DTK+).  Otherwise it can be safely
ignored.

The current parallel implementation of the Schwarz method requires \verb+DTK+,
which is tightly integrated to \trilinos{}, specifically \verb+STK+. Go to the
the top-level \verb+LCM+ directory and create a symbolic link to the \verb+DTK+
CMake fragment that resides in \verb+LCM/Albany/doc/LCM/build+, then download
the \verb+DTK+ package from

\begin{verbatim}
https://github.com/ORNL-CEES/DataTransferKit/archive/2.0.0.tar.gz
\end{verbatim}

\noindent
and place inside the \verb+~/LCM/Trilinos+ directory. Expand the package and
rename it:
\begin{verbatim}
cd ~/LCM
ln -s Albany/doc/LCM/build/dtk-frag.sh .
cd Trilinos
tar zxf 2.0.0.tar.gz
mv DataTransferKit-2.0.0 DataTransferKit
\end{verbatim}
The configuration scripts will detect the presence of \verb+DTK+ and
configure it appropriately. Also, parallel Schwarz will be enabled when
compiling Albany.

\section{Modules}
\label{sec:modules}

Modules are used to create different environments for the
configuration and compilation of both \albany{} and \trilinos{}. To
see the available modules that correspond to different thread models,
compilers and build types:
\begin{verbatim}
module avail
\end{verbatim}
This results in something like:
\begin{verbatim}
------------------ /home/amota/LCM/Albany/doc/LCM/modulefiles ------------------
cuda-gcc-debug       lcm-scientific-linux pthreads-gcc-release
cuda-gcc-mixed       lcm-sems             pthreads-gcc-small
cuda-gcc-profile     lcm-serial           release
cuda-gcc-release     lcm-small            serial-clang-debug
cuda-gcc-small       lcm-tpls             serial-clang-mixed
debug                lcm-ubuntu           serial-clang-profile
lcm-clang            mixed                serial-clang-release
lcm-cluster          openmp-gcc-debug     serial-clang-small
lcm-common           openmp-gcc-mixed     serial-gcc-debug
lcm-cuda             openmp-gcc-profile   serial-gcc-mixed
lcm-debug            openmp-gcc-release   serial-gcc-profile
lcm-fedora           openmp-gcc-small     serial-gcc-release
lcm-finalize         openmp-intel-debug   serial-gcc-small
lcm-gcc              openmp-intel-mixed   serial-intel-debug
lcm-initialize       openmp-intel-profile serial-intel-mixed
lcm-intel            openmp-intel-release serial-intel-profile
lcm-mixed            openmp-intel-small   serial-intel-release
lcm-openmp           profile              serial-intel-small
lcm-profile          pthreads-gcc-debug   small
lcm-pthreads         pthreads-gcc-mixed
lcm-release          pthreads-gcc-profile
\end{verbatim}
The naming convention for the \verb+*-*-*+ modules follows the pattern
\begin{verbatim}
[thread model]-[toolchain]-[build type]
\end{verbatim}
The \verb+[thread model]+ option refers to the thread parallelism
model that the code will use by means of the \kokkos{} package in
\trilinos{}. Currently the sopported models are: \verb+serial+ that
works for all supported compilers, \verb+openmp+ that works with the
\verb+GCC+ and \verb+Intel+ toolchains, \verb+pthreads+ that works for
all supported compilers, and \verb+cuda+ that is only supported for
the \verb+GCC+ toolchain. The installation and configuration of the
Cuda framework is complex. Much more detailed information can be found
at \verb+http://developer.nvidia.com/cuda+.

Currently the options for \verb+[toolchain]+ are \verb+gcc+,
\verb+clang+ and \verb+intel+ if the Intel compilers are installed,
and for \verb+[build type]+ are \verb+debug+ (includes symbolic
information), \verb+release+ (optimization enabled), \verb+profile+
(symbolic information and optimization enabled for profiling),
\verb+small+ (minimizes size of executables) and \verb+mixed+
(\trilinos{} compiled in release mode and \albany{} compiled in debug
mode). The \verb+clang+ toolchain requires installation of the
\verb+clang+ and \verb+clang-devel+ packages. The \verb+debug+,
\verb+release+, \verb+profile+, \verb+small+ and \verb+mixed+ modules
are convenience aliases for \verb+serial-gcc-debug+,
\verb+ serial-gcc-release+, \verb+ serial-gcc-profile+,
\verb+ serial-gcc-small+ and \verb+ serial-gcc-mixed+ modules,
respectively.

Build directories are created within the \lcm{} top-level directory
and named according to the loaded module and package specified to the
build.sh script, e.g.:
\begin{verbatim}
albany-build-gcc-release
\end{verbatim}
In addition, for \trilinos{} an install directory similarly named is
created at the \lcm{} top-level directory.

\section{Configuring and compiling}
\label{sec:config-build}

Assuming that we want to compile with a \verb+serial+ thread model
using the \verb+gcc+ tool chain in \verb+debug+ mode, load the
appropriate module:
\begin{verbatim}
module load serial-gcc-debug
\end{verbatim}

In all these cases one can use the \verb+debug+ convenience alias instead of
\verb+serial-gcc-debug+ for brevity. The \verb+debug+, \verb+release+,
\verb+profile+, \verb+small+ and \verb+mixed+ modules are convenience aliases
for \verb+serial-gcc-debug+, \verb+ serial-gcc-release+,
\verb+ serial-gcc-profile+, \verb+ serial-gcc-small+ and
\verb+ serial-gcc-mixed+ modules, respectively.

Now first configure and compile \trilinos{}. Within the top-level
\verb+LCM+ directory type:
\begin{verbatim}
./clean-config-build.sh trilinos [# processors]
\end{verbatim}
For example, if you want to build using 16 processors, type:
\begin{verbatim}
./clean-config-build.sh trilinos 16
\end{verbatim}
Finally, repeat the procedure for \albany{}:
\begin{verbatim}
./clean-config-build.sh albany [# processors]
\end{verbatim}
For example, if you want to build a version of the code using 16
processors, type:
\begin{verbatim}
./clean-config-build.sh albany 16
\end{verbatim}

Note that to compile a version of \albany{} with a specific thread model,
toolchain and build type, the corresponding version of \trilinos{} must
exist.

\section{After Initial Setup}
The procedure described above configures and compiles the code. From
now on, configuration is no longer required so you can rebuild the
code after any modification by simply using the \verb+build.sh+
script. For example:
\begin{verbatim}
./build.sh albany 16
\end{verbatim}
There are times when it is necessary to reconfigure, for example when
adding or deleting files under the \verb+LCM/Albany/src/LCM+
directory. This is generally announced in the commit notices.

Also, note that both \trilinos{} and \albany{} are heavily templetized
C++ codes. Building the debug version of \albany{} requires large
amounts of memory because of the huge size of the symbolic information
required for debugging. Thus, if the compiling procedure stalls, try
reducing the number of processors.

\section{Running and Debugging \lcm{}} 

After building \albany{}, you might want to run and/or debug the code.
Tools were built in \trilinos{} (decomp, epu, etc.) that are necessary
for parallel execution. The environment created by loading the
appropriate module sets the proper paths so that the executables that
correspond to the type of build are accessible.

\section{Commiting Changes and Code Style}
\albany{} is a simulation code for researchers by researchers. As
such, vibrant development of new and exciting capabilities is strongly
encouraged. For these reasons, don't be afraid to commit changes to
the master git repository. We only ask that you don't break
compilation or testing. So please make sure that the tests pass before
you commit changes. Also, follow the development discussion here:
\begin{verbatim}
https://github.com/gahansen/Albany/issues
\end{verbatim}

In addition, within \lcm{} we strongly encourage you to follow the C++
Google style guide that can be found at
\verb+http://google-styleguide.googlecode.com/svn/trunk/cppguide.html+.
This style is somewhat different to what is currently used in the rest
of \albany{}, but we believe that the Google style is better in that
it advocates more style differentiation between the different
syntactic elements of C++. This in turn makes reading code easier and
helps to avoid coding errors.

The \verb+clang-format+ tool can be used for this. There is a
\verb+.clang-format+ file in the \verb+Albany/src/LCM+ directory that conforms
to the C++ Google coding standard. Thus, all that is needed to reformat a
source file in place is the command:
\begin{verbatim}
clang-format -i [source file name]
\end{verbatim}

\end{document}
